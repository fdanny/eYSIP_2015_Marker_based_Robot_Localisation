\documentclass{article}
\usepackage[a4paper, total={6in, 8in}]{geometry}
\usepackage{color}
\usepackage{verbatim}
\usepackage{listings}

\title{\Huge \sffamily Thresholding}

\author{\LARGE \sffamily Authors: Niharika Jayanthi, Dheeraj Kamath}
\date{ May 31st 2015}

\begin{document}
	\begin{titlepage}
		\maketitle
		\centering
		\textsc{\LARGE Mentor: Sanam Shakya}\\[1.5cm]
	\end{titlepage}
	 
	\begin{titlepage}
		\maketitle
	\end{titlepage}
	

\section*{\centering {\textcolor{blue}{\Huge Thresholding}}}
\Large \sffamily Thresholding is the simplest method of image segmentation. From a grayscale image, thresholding can be used to create binary images. 

\paragraph
{\sffamily \Large What is thresholding? \\ } 
Here, the matter is straight forward. If pixel value is greater than a threshold value, it is assigned one value (may be white), else it is assigned another value (may be black). The function used is cv2.threshold. First argument is the source image, which should be a grayscale image. Second argument is the threshold value which is used to classify the pixel values. Third argument is the maxVal which represents the value to be given if pixel value is more than (sometimes less than) the threshold value. OpenCV provides different styles of thresholding and it is decided by the fourth parameter of the function. Different types are:
\begin{itemize}
\item  \begin{verbatim} cv2.THRESH_BINARY \end{verbatim}
\item  \begin{verbatim} cv2.THRESH_BINARY_INV \end{verbatim}
\item  \begin{verbatim} cv2.THRESH_TOZERO \end{verbatim}
\item  \begin{verbatim} cv2.THRESH_TRUNC \end{verbatim}
\item  \begin{verbatim} cv2.THRESH_TOZERO_INV \end{verbatim}
\end{itemize}

\section*{\sffamily How to use the function?}
\sffamily The function takes three inputs. First is the source image(img),
second argument is the threshold value which is used to classify the pixel values.
Third argument is the \verb Max_value  which represents the value to be given if pixel value
is more than (sometimes less than) the threshold value.
 \begin{lstlisting} 
 ret,thresh = cv2.threshold(img, Min_value, Max_value,x) 
 where x is any of the five types of threshold.
 \end{lstlisting} 
\end{document}